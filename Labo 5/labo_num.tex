\documentclass[11pt,a4paper]{article}
\usepackage[utf8x]{inputenc}
\usepackage[T1]{fontenc}
\usepackage{ucs}
\usepackage{amsthm} %numéroter les questions
\usepackage[frenchb]{babel}
\usepackage{datetime}
\usepackage{xspace} % typographie IN
\usepackage{hyperref}% hyperliens
\usepackage[all]{hypcap} %lien pointe en haut des figures
\usepackage[french]{varioref} %voir x p y
\usepackage{fancyhdr}% en têtes
%\input cyracc.def
\usepackage[]{graphicx} %include pictures
\usepackage{pgfplots}
\usetikzlibrary{babel,positioning,calc}
\usepackage[americanresistors ]{circuitikz}
\usepackage[]{gnuplottex}
\usepackage{ifthen}
\usepackage{mathastext} % math as standfard text : units are respecting typography conventions.

\usepackage[top=1.3 in, bottom=1.3 in, left=1.3 in, right=1.3 in]{geometry} % Yeah, that's bad to play with margins
\usepackage[]{pdfpages}
\usepackage[]{subfig}
\usepackage[]{attachfile}
%\usepackage{ stmaryrd } % for \ligthning

\newdateformat{mydate}{}%hack pour remplacer \THEYEAR

%cyr
\newcommand\textcyr[1]{{\fontencoding{OT2}\fontfamily{wncyr}\selectfont #1}}


\newboolean{corrige}
\ifx\correction\undefined
\setboolean{corrige}{false}% pas de corrigé
\else
\setboolean{corrige}{true}%corrigé
\fi

%\setboolean{corrige}{false}% pas de corrigé

\newboolean{annexes}
\setboolean{annexes}{true}%annexes
%\setboolean{annexes}{false}% pas de annexes

\definecolor{darkblue}{rgb}{0,0,0.5}

\usepackage{aeguill} %guillemets

%% fancy header & foot

\lhead{[ELEC-H-301] Électronique appliquée\\ LABO \no 5 : circuits logiques}
\rhead{v1.0.0 \\ page \thepage}
\chead{\ifthenelse{\boolean{corrige}}{Corrigé}{}}
\cfoot{}
%%
\pagestyle{fancy}

\pdfinfo{
/Author (ULB -- BEAMS)
/Title (Labo n° 5 ELEC-H-301, circuits logiques)
/ModDate (D:\pdfdate)
}

\hypersetup{
pdftitle={Labo n° 5 [ELEC-H-301] Électronique appliquée : circuits logiques},
pdfauthor={Yannick Allard, ©2016 ULB - BEAMS  },
pdfsubject={Diodes et alimentations}
}

\theoremstyle{definition}% questions pas en italique
\newtheorem{Q}{Question}[] % numéroter les questions [section] ou non []

\newcommand{\reponse}[1]{% pour intégrer une réponse : \reponse{texte} : sera inclus si \boolean{corrige}
	\ifthenelse {\boolean{corrige}} {\paragraph{Réponse :} \color{darkblue}   #1\color{black}} {}
 }

\fancyhfoffset[ERLO]{0cm} %header with full page width

\newcommand{\addcontentslinenono}[4]{\addtocontents{#1}{\protect\contentsline{#2}{#3}{#4}{}}}

\date{}
\title{Séance 5~: Circuits logiques\ifthenelse{\boolean{corrige}}{~\\Corrigé}{}}

%\author{\vspace{-1cm}}%\textsc{Yannick Allard}}

\setlength{\parskip}{0.2cm plus2mm minus1mm} %espacement entre §
\setlength{\parindent}{0pt}
\setlength{\headheight}{52pt}
\fancypagestyle{plain}

\newcommand{\itgv}[1]{\ifthenelse{\boolean{corrige}}{{\color{blue}#1}}{}} %si corrigé vrai...
\newcommand{\ifgv}[1]{\ifthenelse{\boolean{corrige}}{}{#1}} %si corrigé vrai...


%from SO: kinky cross for wires
\tikzset{
  declare function={% in case of CVS which switches the arguments of atan2
    atan3(\a,\b)=ifthenelse(atan2(0,1)==90, atan2(\a,\b), atan2(\b,\a));},
  kinky cross radius/.initial=+.125cm,
  @kinky cross/.initial=+, kinky crosses/.is choice,
  kinky crosses/left/.style={@kinky cross=-},kinky crosses/right/.style={@kinky cross=+},
  kinky cross/.style args={(#1)--(#2)}{
    to path={
      let \p{@kc@}=($(\tikztotarget)-(\tikztostart)$),
          \n{@kc@}={atan3(\p{@kc@})+180} in
      -- ($(intersection of \tikztostart--{\tikztotarget} and #1--#2)!%
             \pgfkeysvalueof{/tikz/kinky cross radius}!(\tikztostart)$)
      arc [ radius     =\pgfkeysvalueof{/tikz/kinky cross radius},
            start angle=\n{@kc@},
            delta angle=\pgfkeysvalueof{/tikz/@kinky cross}180 ]
      -- (\tikztotarget)}}}


\begin{document}
\pagestyle{fancy}
\maketitle
\vspace*{-1cm}
\section{Introduction}

\subsection{But}
Cette manipulation a pour but d'illustrer :
\begin{itemize}
\item au niveau «application» : le fonctionnement de circuits logiques câblés
\item au niveau «composant» : le fonctionnement de circuits logiques discrets.
\end{itemize}

\subsection{Prérequis}
\marginpar{\color{white}"We will rise again" Peia}
%\begin{itemize}
%\item algèbre de Boole
%\item
Chapitres \no 24 et \no 25  du livre de référence (ed 5).\\ En particulier :
\begin{itemize}
\item algèbre de Boole
\item portes logiques, états logiques
\item logique combinatoire
\item délai de propagation
\item logique séquentielle : bascules R-S et D.
\end{itemize}

\textbf{Veillez aussi à relire l'annexe du laboratoire \no 1 sur les instruments de laboratoire.}

%\end{itemize}

\subsection{Prédéterminations}
Les questions~\ref{Q:tv1},~\ref{Q:tv2},~\ref{Q:chrono},~\ref{Q:rs_deb} à ~\ref{Q:rs_fin}, ~\ref{Q:dl1} et~\ref{Q:dlatch} doivent être faites \textbf{avant} l'arrivée au laboratoire.

%Le TP 4 portant sur les diodes fait également office de prédéterminations.
%Les prédéterminations des questions \ref{Q:1} à \ref{Q:predet}


\subsection{Objectifs}
À la fin de ce laboratoire, vous devez être capable de :
\begin{itemize}
\item donner la table de vérité d'un circuit logique
\item réaliser et tester un circuit logique
\item comprendre la notion de temps de propagation et le mesurer à l'oscilloscope
\item comprendre le fonctionnement d'une bascule R-S
\item réaliser et tester une bascule R-S et une bascule D à l'aide de portes logiques.
\end{itemize}
%\clearpage
\newpage
\pagestyle{fancy}
%\begin{Q}
%	À quelle condition cette source est-elle linéaire ?
%	\label{Q:1}
%	\reponse{$g$ doit être constant.}
%\end{Q}

\section{Concepts}

Les circuits logiques sont invisibles mais fortement présent autour de nous : ordinateurs, téléphones, tablettes numériques\dots Même si la complexité des circuits a fortement augmenté depuis quelques décennies, les fonctions de base sont restées essentiellement identiques. Ce labo a pour but de vous montrer comment réaliser les fonctions de base de la logique numérique avec des circuits conçus à cet effet.

\section{Logique combinatoire}

Les circuits décrits dans cette section sont \textbf{combinatoires}, c'est à dire que leur sortie ne dépend que de l'état des entrées.

%\subsection{Opérations de base}
Les opérations logiques usuelles\footnote{La représentation américaine des portes logiques est utilisée ici, une représentation européenne existe également} sont :

% rappel des portes logiques
%footnote : représentation US

\begin{center}
%\shorthandoff{:!}
		\begin{circuitikz} \draw
		(0,0) node [american nand port]{}
		(-0.7,-0.8) node  {NAND}

		(2,0) node [american nor port] {}
		(2-0.7,-0.8) node  {NOR}

		(4,0) node [american xnor port] {}
		(4-0.7,-0.8) node  {XNOR}

		(0,2) node [american and port] {}
		(-0.7,2-0.8) node  {AND}

		(2,2) node [american or port] {}
		(2-0.7,2-0.8) node  {OR}

		(4,2) node [american xor port] {}
		(4-0.7,2-0.8) node  {XOR}

		(6,1) node [american not port] {}
		(6.7-0.7,1-0.8) node  {NOT}

	;\end{circuitikz}
%\shorthandon{:!}
	\end{center}

\begin{Q}
	Donner la table de vérité du circuit suivant :

	\begin{center}
%\shorthandoff{:!}
		\begin{circuitikz} \draw
		(0,0) node [american nor port] (nor) {}
%		(-0.7,-0.8) node  {NAND}
		(nor.in 2) -- ++(-1,0) node [ocirc] {} node [anchor=east] {C}
		(nor.in 1) -|  (-1.5,1)

		(0,2) node [american and port] (and){}
%		(0-0.7,2-0.8) node  {NOR}
		(and.in 2) -| (-1.5,1) to [short, -o] (-2.4,1) node [anchor=east] {B}
		(and.in 1)	-- ++(-1,0) node [ocirc] {} node [anchor=east] {A}

		(2,1) node [american or port] (or){}
%		(4-0.7,1-0.8) node  {XNOR}
		(and.out) -| (or.in 1)
		(nor.out) -| (or.in 2)

		(3.5,1) node [american not port] (not){}
		(or.out) -- (not.in)
		(not.out) -- ++(1,0) node [ocirc] {} node [anchor=west] {X}


%		(0,2) node [american and port] {}
%		(-0.7,2-0.8) node  {AND}

%		(2,2) node [american or port] {}
%		(2-0.7,2-0.8) node  {OR}

%		(4,2) node [american xor port] {}
%		(4-0.7,2-0.8) node  {XOR}


%		(6.7-0.7,1-0.8) node  {NOT}

	;\end{circuitikz}
%\shorthandon{:!}
	\end{center}
	\label{Q:tv1}
	%circuit
	\reponse{
	\begin{tabular}{ccc|c}
		A & B & C & X \\
		\hline
		0 & 0 & 0 & 0 \\
		0 & 0 & 1 & 1 \\
		0 & 1 & 0 & 1 \\
		0 & 1 & 1 & 1 \\
		1 & 0 & 0 & 0 \\
		1 & 0 & 1 & 1 \\
		1 & 1 & 0 & 0 \\
		1 & 1 & 1 & 0 \\
		\end{tabular}

	}%R
\end{Q}

%\subsection{Exercices}
\begin{Q}
	En utilisant une table de vérité, démontrer que ce circuit est un «voteur» :
	%circuit
	\begin{center}
%\shorthandoff{:!}
		\begin{circuitikz} \draw
		(0,0) node [american and port] (and1) {}
%		(-0.7,-0.8) node  {NAND}
		(and1.in 2) -- ++(-0.5,0) |- node [circ] {} ++(-0.5,2.56) node [ocirc] (B) {} node [anchor=east] {B}


		(0,2) node [american or port] (or){}
%		(0-0.7,2-0.8) node  {NOR}
		(or.in 1) -- ++(-0.5,0) |- (B)
		(or.in 2) |- node [circ] {} ++(-1,-0.4) node [ocirc] (C) {} node [anchor=east] {C}
		(and1.in 1) |- (C)

		(2,3) node [american and port] (and2) {}
%		(4-0.7,1-0.8) node  {XNOR}
		(or.out) -| (and2.in 2)
		(and2.in 1) -- ++(-3,0)  node [ocirc] (A) {} node [anchor=east] {A}
		(or.out) -| (and2.in 2)

		(3.5,1) node [american or port] (and3){}
		(and2.out) -| (and3.in 1)
		(and1.out) -| (and3.in 2)
		(and3.out) -- ++(1,0) node [ocirc] {} node [anchor=west] {Y}


%		(0,2) node [american and port] {}
%		(-0.7,2-0.8) node  {AND}

%		(2,2) node [american or port] {}
%		(2-0.7,2-0.8) node  {OR}

%		(4,2) node [american xor port] {}
%		(4-0.7,2-0.8) node  {XOR}


%		(6.7-0.7,1-0.8) node  {NOT}

	;\end{circuitikz}
%\shorthandon{:!}
	\end{center}
	\label{Q:tv2}
	\reponse{
	\begin{tabular}{ccc|c}
		A & B & C & X \\
		\hline
		0 & 0 & 0 & 0 \\
		0 & 0 & 1 & 0 \\
		0 & 1 & 0 & 0 \\
		0 & 1 & 1 & 1 \\
		1 & 0 & 0 & 0 \\
		1 & 0 & 1 & 1 \\
		1 & 1 & 0 & 1 \\
		1 & 1 & 1 & 1 \\
		\end{tabular}
		}%R
\end{Q}

\begin{Q}
	Réaliser ce circuit sur le protoboard au moyen de circuits intégrés 74HC00 (\textbf{N}AND) et 74HC32 (OR), les brochages se trouvent en annexe.
	\begin{itemize}
	\item Brancher les entrées sur les leds pour visualiser leur état,
	\item brancher les sorties intermédiaires sur les leds pour vérifier progressivement votre câblage,
	\item brancher la sortie sur une led pour vérifier son état.
	\item \textbf{N'oubliez pas d'alimenter vos circuits logiques en 5V !}
	\end{itemize}

	\textit{Note~: Réalisez d'abord un circuit permettant de simuler le comportement d'une porte AND à l'aide de deux portes NAND.}
	\label{Q:2}
	\reponse{}%R
\end{Q}

\begin{Q}
	Tester son bon fonctionnement à l'aide des LEDs et des interrupteurs. Vérifier si sa table de vérité correspond à celle déterminée précédemment.
	\label{Q:1}
	\reponse{}%R
\end{Q}

\begin{Q}
	Compléter le chronogramme ci-dessous en considérant que la porte "NOT" a un temps de propagation de 10ns et que la porte "AND" a un temps de propagation de 20 ns.
		\begin{center}
%\shorthandoff{:!}
		\begin{circuitikz} \draw
		(0,0.72) node [american and port] (and1) {}
%		(-0.7,-0.8) node  {NAND}
		(-3,1) node [american not port, scale=0.8] (not){}
		(and1.in 1) -|   (not.out)

		(not.in) |-  ++(-0.5,0) node [ocirc] (A) {} node [anchor=east] {A}
		(and1.in 2) |- ++(-2.66,0) node [ocirc] (B) {} node [anchor=east] {B}
		(and1.out) -- ++(1,0) node [ocirc] (Y) {} node [anchor=west] {Y}

	;\end{circuitikz}
%\shorthandon{:!}
%\shorthandoff{:!}

\begin{tikzpicture}[minimum size= 1.25cm,scale=0.9]
%\usetikzlibrary{arrows,decorations.markings}
%\def \n {5}
%\def \radius {3cm}
%\def \margin {8} % margin in angles, depends on the radius
\usetikzlibrary{calc}
%\usetikzlibrary{arrows.meta}
%\foreach \s in {1,...,\n}
{

%\tikzset{>={Latex[width=3mm,length=3mm]}}
%/pgf/;
	\draw   [->]  (0,0) -- (0,1);
  \node [anchor=east] at (0,1) {A};
  \draw [->]( 0,0) -- (10.5,0);
  \node [anchor=west] at (10.5,0) {t};

  \foreach \x in {1,2,...,10} \draw (\x,-0.1) -- (\x,0.1);
  \foreach \x in {1,2,...,10} \draw (\x,-0.1-2) -- (\x,0.1-2);
  \foreach \x in {1,2,...,10} \draw (\x,-0.1-4) -- (\x,0.1-4);
  \node [anchor=north, inner sep=0pt, outer sep=0pt] at (1,0.25) {10ns};
  \node [anchor=north, inner sep=0pt, outer sep=0pt] at (2,0.25) {20ns};

  	\draw [->] (0,-2) -- (0,1-2);
  \node [anchor=east] at (0,1-2) {B};
  \draw [->] (0,-2) -- (10.5,-2);
  \node [anchor=west] at (10.5,-2) {t};

  	\draw [->] (0,-4) -- (0,1-4);
  \node [anchor=east] at (0,1-4) {Y};
  \draw [->] (0,-4) -- (10.5,-4);
  \node [anchor=west] at (10.5,-4) {t};

  \draw [line width=2pt] (0,0) -|(3,1) -| (10,1); %A
   \draw [line width=2pt] (0,0-2) -|(1,1-2) -| (5,0-2) -- (10,0-2); %B

   \itgv{
   \draw [line width=2pt, color=darkblue] (0,-4) -| (6,1-4) -|(7,0-4)--(10,0-4); %Y
   }

}
\end{tikzpicture}
	\label{Q:chrono}
	\end{center}
	%diagramme
	\label{Q:3}
	\reponse{}%R
\end{Q}

\begin{Q}
	Mettre en évidence le temps de propagation sur le circuit \textit{de la question~\ref{Q:tv2}}.
	\begin{itemize}
	\item Brancher $A$ sur l'entrée de synchronisation de l'oscilloscope et l'utiliser pour le déclenchement de l'oscilloscope,
	\item brancher la sortie de $A\cdot(B+C)$ sur \verb!CH1!,
	\item brancher $Y$ sur \verb!CH2!,
	\item utiliser les curseurs et le retard par rapport au déclenchement pour déterminer les temps de propagation.
	\end{itemize}%Utiliser l'entrée de synchronisation de l'oscilloscope comme référence.
	\label{Q:1}
	\reponse{}%R
\end{Q}

\newpage
\section{Logique séquentielle, mémorisation}

Les bistables sont des circuits très employés en électronique numérique en raison de leur multiples applications. Leur première fonction est de mémoriser une information logique.

Les bistables sont des circuits \textbf{séquentiels}, c'est à dire que leur sortie dépend des entrées \textbf{et} de l'état précédent du système. Contrairement aux circuits combinatoires de la section précédente, l'état actuel du système dépend de son passé.

\subsection{Bistable élémentaire}
Ci-dessous est représenté sous 2 formes différentes le bistable le plus simple.


\begin{center}
		\begin{circuitikz} \draw
		(0,0) node [american not port] (not3) {}
		(2,0) node [american not port] (not4) {}
		(not3.out) -- (not4.in)
		(not4.out) -- ++(0.5,0) |- ++(-4,-1) |- (not3.in)
		(not3.out) |-  ++(0.25,0) node [circ] () {} node [anchor=south] {$\overline{Q}$}
		(not4.out) |-  ++(0.25,0) node [circ] () {} node [anchor=south] {$Q$}

		(8,1) node [american not port] (not1) {}
		(8,-1) node [american not port] (not2) {}
%		(not1.out) -- (not2.in)
	%	(not2.out) -| ++(0.5,0.5)  coordinate (a-a) to [kinky cross=(a-a)--(a-b), kinky crosses=left] ++(-3,1)  coordinate (a-b) |- (not1.in)
		(not1.out)  ++(0.5,-0.5)  coordinate (a-a) %coords of the crossing wire
		(not2.in)  ++(-1,0.5)  coordinate (a-b)

		(not1.in)++(-1.27,-0.5) node (in) {} % end of the wire with kinky bump
		(not2.out)-| ++(0.5,0.5) to  [kinky cross=(a-a)--(a-b), kinky crosses=left] (in)
		(not1.in)-| ++(-1.14,-0.55)
		(not2.out) -- ++(1,0) node [circ] () {} node [anchor=south] {$\overline{Q}$}

		(not1.out) -| ++(0.5,-0.5) -- ++(-3.043,-1) |- (not2.in)
		(not1.out) -- ++(1,0) node [circ] () {} node [anchor=south] {$Q$}
%		(not1.out) |-  ++(0.25,0)
%		(not2.out) |-  ++(0.25,0) node [circ] () {} node [anchor=south] {$Q$}

	;\end{circuitikz}
\end{center}
%schéma

%\newpage
Ce bistable possède 2 états stables (d'où son nom) :
\begin{itemize}
\item soit $Q$ est à l'état haut (1) et $\overline{Q}$ est à l'état bas (0)
\item soit $Q$ est à l'état bas  (0) et $\overline{Q}$ est à l'état haut (1)
\end{itemize}

Ce circuit est donc une mémoire. Malheureusement, il est difficile de modifier son état autrement
qu'en court-circuitant une des sortie à un état déterminé: il n'y a en effet pas d'entrée à ce circuit.

\subsection{Amélioration}
Imaginons que l'on dispose d'inverseurs dont on peut mettre la sortie dans un état déterminé. Par
exemple l'inverseur suivant :

%schéma

\begin{center}
\begin{circuitikz} \draw

		(0,0) node [american not port] (not1) {}
		(not1)+(0,.25) |- ++(-1.7,1) node [ocirc] () {} node [anchor=south] {$B$}
		(not1.in) -- ++(-1,0) node [ocirc] () {} node [anchor=south] {$A$}
		(not1.out) -- ++(1,0) node [ocirc] () {} node [anchor=south] {$Q$}

	;\end{circuitikz}

\end{center}
\begin{itemize}
\item si B est à l'état bas, alors Q est l'inverse de A (inversion normale de A)
\item si B est à l'état haut, alors Q est à l'état bas (par exemple)
\end{itemize}

On peut alors réaliser le montage suivant :

%schéma
\begin{center}
\begin{circuitikz} \draw
		(0,0) node [american not port] (not3) {}
		(not3)+(0,.25) |- ++(-0.7,1) node [ocirc] () {} node [anchor=south] {$B1$}
		(2,0) node [american not port] (not4) {}
		(not4)+(0,.25) |- ++(-0.7,1) node [ocirc] () {} node [anchor=south] {$B2$}
		(not3.out) -- (not4.in)
		(not4.out) -- ++(0.5,0) |- ++(-4,-1) |- (not3.in)
		(not3.out) |-  ++(0.25,0) node [circ] () {} node [anchor=south] {$\overline{Q}$}
		(not4.out) |-  ++(0.25,0) node [circ] () {} node [anchor=south] {$Q$}

		(8,1) node [american not port] (not1) {}
		(not1)+(0,0.25) |- ++(-0.7,1) node [ocirc] () {} node [anchor=south] {$B2$}
		(8,-1) node [american not port] (not2) {}
		(not2)+(0,-0.25) |- ++(-0.7,-1) node [ocirc] () {} node [anchor=north] {$B1$}
%		(not1.out) -- (not2.in)
	%	(not2.out) -| ++(0.5,0.5)  coordinate (a-a) to [kinky cross=(a-a)--(a-b), kinky crosses=left] ++(-3,1)  coordinate (a-b) |- (not1.in)
		(not1.out)  ++(0.5,-0.5)  coordinate (a-a) %coords of the crossing wire
		(not2.in)  ++(-1,0.5)  coordinate (a-b)

		(not1.in)++(-1.27,-0.5) node (in) {} % end of the wire with kinky bump
		(not2.out)-| ++(0.5,0.5) to  [kinky cross=(a-a)--(a-b), kinky crosses=left] (in)
		(not1.in)-| ++(-1.14,-0.55)
		(not2.out) -- ++(1,0) node [ocirc] () {} node [anchor=south] {$\overline{Q}$}

		(not1.out) -| ++(0.5,-0.5) -- ++(-3.043,-1) |- (not2.in)
		(not1.out) -- ++(1,0) node [ocirc] () {} node [anchor=south] {$Q$}
%		(not1.out) |-  ++(0.25,0)
%		(not2.out) |-  ++(0.25,0) node [circ] () {} node [anchor=south] {$Q$}

	;\end{circuitikz}
\end{center}

On peut maintenant réaliser les opérations suivantes :
\begin{itemize}
\item mettre l'entrée B2 à l'état haut et l'entrée B1 à l'état bas : la sortie Q est alors mise à l'état bas.
\item mettre l'entrée B2 à l'état bas et l'entrée B1 à l'état haut : la sortie Q est alors mise à l'état haut.
\item  mettre les 2 entrées B1 et B2 à l'état bas : le circuit reste dans sont état précédant : c'est la
mémorisation.
\end{itemize}


Le cas où les 2 entrées B1 et B2 sont à l'état haut n'est pas intéressant : il force les 2 sorties $Q$ et $\overline{Q}$ à
l'état bas. C'est un cas indésirable.
\subsection{Bistable RS}
En fait, on connaît, sans le savoir, des éléments logiques qui répondent à l'inverseur spécial vu ci-dessus.


Prenons par exemple la porte NOR (Non OU).

%schéma

\begin{center}
\begin{circuitikz} \draw

		(0,0) node [american nor port] (nor) {}
%		(not1)+(0,.25) |- ++(-1.7,1) node [ocirc] () {} node [anchor=south] {$B$}
		(nor.in 1) -- ++(-1,0) node [ocirc] () {} node [anchor=east] {$A$}
		(nor.in 2) -- ++(-1,0) node [ocirc] () {} node [anchor=east] {$B$}
		(nor.out) -- ++(1,0) node [ocirc] () {} node [anchor=west] {$Q$}

	;\end{circuitikz}

\end{center}

Cette porte répond parfaitement à la table de vérité de l'inverseur spécial : si son entrée B est à l'état
haut, la sortie Q est forcée à l'état bas ; sinon, la sortie Q est l'inverse de l'entrée A.
Construisons alors le bistable à partir de ces portes et renommons (S, R) les entrées (B1, B2).

%schéma

\begin{center}
\begin{circuitikz} \draw
		(0,1) node [american nor port] (nor1) {}
	%	(nor1)+(0,0.25) |- ++(-0.7,1) node [ocirc] () {} node [anchor=south] {$B2$}
		(0,-1.5) node [american nor port] (nor2) {}
	%	(nor2)+(0,-0.25) |- ++(-0.7,-1) node [ocirc] () {} node [anchor=north] {$B1$}
%		(not1.out) -- (not2.in)
	%	(not2.out) -| ++(0.5,0.5)  coordinate (a-a) to [kinky cross=(a-a)--(a-b), kinky crosses=left] ++(-3,1)  coordinate (a-b) |- (not1.in)
		(nor1.out)  ++(0.5,-0.5)  coordinate (a-a) %coords of the crossing wire
		(nor2.in 2)  ++(-1.5,0.5)  coordinate (a-b)

		(nor1.in 2)++(-1.135,-0.225) node (in) {} % end of the wire with kinky bump
		(nor2.out)-| ++(0.5,0.5) to  [kinky cross=(a-a)--(a-b), kinky crosses=left] (in)
		(nor1.in 2)-| ++(-1,-0.3)
		(nor1.out) -- ++(1.5,0) node [ocirc] () {} node [anchor=west] {$\overline{Q}$}

		(nor1.out) -| ++(0.5,-0.5) -- ++(-3.043,-1.5) |- (nor2.in 1)
		(nor2.out) -- ++(1.5,0) node [ocirc] () {} node [anchor=west] {$Q$}
		(nor1.out) |-  ++(0.25,0)
		%(nor2.out) |-  ++(0.25,0) node [circ] () {} node [anchor=south] {$Q$}

		(nor1.in 1) -- ++(-2,0) node [ocirc] () {} node [anchor=east] {$S$}
		(nor2.in 2) -- ++(-2,0) node [ocirc] () {} node [anchor=east] {$R$}

	;
	\draw [dashed](-2.75,-2.25) rectangle (1,1.75);
	\end{circuitikz}
\end{center}

\begin{itemize}
\item Pour mettre la sortie Q à l'état bas, il suffit d'appliquer un état haut à l'entrée R (Reset : mise à l'état bas)
et un état bas à l'entrée S (Set : mise à l'état haut)
\item Pour mettre la sortie Q à l'état haut, il suffit d'appliquer un état haut à l'entrée S (Set : mise à l'état haut)
et un état bas à l'entrée R (Reset : mise à l'état bas)
\item Pour être en mémorisation, il suffit d'appliquer un état bas aux 2 entrées.
\end{itemize}


Si on appelle l'état bas le niveau inactif et l'état haut le niveau actif, alors on peut dire :
\begin{itemize}
\item pour mettre le bistable dans un état déterminé, il suffit d'appliquer un niveau actif à l'entrée
correspondante.
\item pour mettre le bistable en mémorisation, il suffit de ne mettre aucun niveau actif
\item l'état indésirable a lieu lorsqu'on demande au bistable en même temps une mise à l'état haut et une mise
à l'état bas.
\end{itemize}

\subsection{Table de vérité du bistable R-S}
Si on n'est pas convaincu par les développements précédents, on peut écrire la table de vérité du
bistable R-S :

%table
\begin{center}
\begin{tabular}{cccc|cc}
R & S & $Q_n$ & $\overline{Q_n}$ & $Q_{n+1}$ & $\overline{Q_{n+1}}$ \\
\hline
0 & 0 & 0 & 0 & ? & ? \\
0 & 0 & 0 & 1 & 0 & 1 \\
0 & 0 & 1 & 0 & 1 & 0 \\
0 & 0 & 1 & 1 &  \multicolumn{2}{c}{impossible} \\
0 & 1 & 0 & 0 & 1 & 0 \\
0 & 1 & 0 & 1 & 1 & 0 \\
0 & 1 & 1 & 0 & 1 & 0 \\
0 & 1 & 1 & 1 & \multicolumn{2}{c}{impossible}  \\
1 & 0 & 0 & 0 & 0 & 1 \\
1 & 0 & 0 & 1 & 0 & 1 \\
1 & 0 & 1 & 0 & 0 & 1 \\
1 & 0 & 1 & 1 & \multicolumn{2}{c}{impossible}  \\
1 & 1 & 0 & 0 & 0 & 0 \\
1 & 1 & 0 & 1 & 0 & 0 \\
1 & 1 & 1 & 0 & 0 & 0 \\
1 & 1 & 1 & 1 & \multicolumn{2}{c}{impossible}  \\
\end{tabular}
\end{center}

En entrée de la table, on met les 2 sorties stables à l'instant t et les entrée que l'on applique à ce
même instant. En sortie de la table, on détermine l'état des 2 sorties à l'instant $t+t_p$, c'est-à-dire après le
temps de propagation du bistable.
Cette table de vérité peut être simplifiée, en effet, les états impossibles ne sont pas à prendre en
considération puisqu'ils ne seront jamais rencontrés. Il ne faut cependant pas confondre état
impossible en entrée (lignes impossibles) et état indéterminé en sortie (ligne 1). La table devient:


%table
\begin{center}
\begin{tabular}{cccc|cc}
R & S & $Q_n$ & $\overline{Q_n}$ & $Q_{n+1}$ & $\overline{Q_{n+1}}$ \\
\hline
0 & 0 & 0 & 0 & ? & ? \\
0 & 0 & 0 & 1 & 0 & 1 \\
0 & 0 & 1 & 0 & 1 & 0 \\
0 & 1 & X & X & 1 & 0 \\
1 & 0 & X & X & 0 & 1 \\
1 & 1 & X & X & 0 & 0 \\
\end{tabular}
\end{center}

Le dernier état est évidemment indésirable puisqu'il rend égales des sorties normalement
complémentaires (l'ordre donné au bistable est incohérent). Si on évite cet état, on peut supprimer la
dernière ligne du tableau ainsi que la première. On peut alors simplifier cette table pour avoir :


%table
\begin{center}
\begin{tabular}{cc|cc}
R & S & $Q_{n+1}$ & $\overline{Q_{n+1}}$ \\
\hline & & & \\[-1em]
0 & 0 & $Q_n$ & $\overline{Q_n}$ \\
0 & 1 & 1 & 0 \\
1 & 0 & 0 & 1 \\
\end{tabular}
\end{center}


De cette table, on retire heureusement les mêmes conclusions que précédemment.
La méthode par table de vérité est assez lourde dans son entièreté. Il vaut donc mieux essayer
d'établir directement la table simplifiée comme vu précédemment.

\subsection{Réalisation du bistable R-S}
%Réaliser le schéma suivant à l'aide de portes logiques :
Soit le circuit suivant :

%schéma

\begin{center}
\begin{circuitikz} \draw
		(0,1) node [american nand port] (nor1) {}
	%	(nor1)+(0,0.25) |- ++(-0.7,1) node [ocirc] () {} node [anchor=south] {$B2$}
		(0,-1.5) node [american nand port] (nor2) {}
	%	(nor2)+(0,-0.25) |- ++(-0.7,-1) node [ocirc] () {} node [anchor=north] {$B1$}
%		(not1.out) -- (not2.in)
	%	(not2.out) -| ++(0.5,0.5)  coordinate (a-a) to [kinky cross=(a-a)--(a-b), kinky crosses=left] ++(-3,1)  coordinate (a-b) |- (not1.in)
		(nor1.out)  ++(0.5,-0.5)  coordinate (a-a) %coords of the crossing wire
		(nor2.in 2)  ++(-1.5,0.5)  coordinate (a-b)

		(nor1.in 2)++(-1.135,-0.225) node (in) {} % end of the wire with kinky bump
		(nor2.out)-| ++(0.5,0.5) to  [kinky cross=(a-a)--(a-b), kinky crosses=left] (in)
		(nor1.in 2)-| ++(-1,-0.3)
		(nor1.out) -- ++(1.5,0) node [ocirc] () {} node [anchor=west] {$Q$}

		(nor1.out) -| ++(0.5,-0.5) -- ++(-3.043,-1.5) |- (nor2.in 1)
		(nor2.out) -- ++(1.5,0) node [ocirc] () {} node [anchor=west] {$\overline{Q}$}
		(nor1.out) |-  ++(0.25,0)
		%(nor2.out) |-  ++(0.25,0) node [circ] () {} node [anchor=south] {$Q$}

		(nor1.in 1) -- ++(-2,0) node [ocirc] () {} node [anchor=east] {$\overline{S}$}
		(nor2.in 2) -- ++(-2,0) node [ocirc] () {} node [anchor=east] {$\overline{R}$}

	;
	\draw [dashed](-2.75,-2.25) rectangle (1,1.75);
	\end{circuitikz}
\end{center}

\begin{Q}
	Quel est l'état des entrées qui assure la fonction de mémorisation ?
	\label{Q:rs_deb}
	\reponse{}%R
\end{Q}

\begin{Q}
	Que faut il appliquer au circuit pour réaliser un Set ou Reset ?
	\label{Q:1}
	\reponse{}%R
\end{Q}

\begin{Q}
	Quelle est l'anomalie des sorties lorsque les deux entrées sont à l'état bas ?
	\label{Q:1}
	\reponse{}%R
\end{Q}


\begin{Q}
	Donner à chaque valeur des entrées ($\overline{R}$, $\overline{S}$) une appellation parmi les suivantes :
	\begin{itemize}
	\item mémorisation
	\item mise à 0
	\item mise à 1
	\item indésirable
	\end{itemize}
	\label{Q:1}
	\reponse{}%R
\end{Q}

\begin{Q}
	Justifier appellation R-S. (R : Reset S : Set); Quel est le niveau logique actif de ces 2 entrées ?
	\label{Q:1}
	\reponse{}%R
\end{Q}

\begin{Q}
	Pourquoi parle-t-on d'entrées en logique inverse ?
	\label{Q:1}
	\reponse{}%R
\end{Q}
\begin{Q}
	Que pourrait-il se passer si on passe de l'état [mise à 1] à l'état [mémorisation] en passant transitoirement
par l'état [indésirable] ?
	\label{Q:rs_fin}
	\reponse{}%R
\end{Q}

\begin{Q}
	Réaliser le circuit ci-dessus sur le protoboard et vérifier vos conclusions des questions~\ref{Q:rs_deb} à~\ref{Q:rs_fin}.
	\label{Q:1}
	\reponse{}%R
\end{Q}

%\begin{Q} % on peut pasremplacer directement le 74HC00 par un HC00, mais on peut pas un HC36, introuvable...
%	Remplacer les portes NAND par des portes NOR (remplacer le circuit). Comment le comportement est-il affecté ?
%	\label{Q:1}
%	\reponse{}%R
%\end{Q}
%
%\begin{Q}
%	Pour la suite du labo, revenir au circuit basé sur des portes NAND.
%	\label{Q:1}
%	\reponse{}%R
%\end{Q}

%FIXME TODO style pour les états

\subsection{Bistable R-S avec entrée d'activation (Enable)}

Modifier le schéma précédent comme suit : %, \textbf{en vérifiant bien que les bonnes portes logiques sont utilisées} :

%schéma

\begin{center}
\begin{circuitikz} \draw
		(0,1.28) node [american nand port] (nand1) {}
		(0,-1.5-0.28) node [american nand port] (nand2) {}

		(nand1.in 1) -- ++(-1.5,0) node [ocirc] () {} node [anchor=east] {$R$}
		(nand2.in 2) -- ++(-1.5,0) node [ocirc] () {} node [anchor=east] {$S$}

%		(nand1.in 2)
		(nand1.in 2) |- ++(-1.5,-1.28) coordinate (dot) node [ocirc] () {} node [anchor=east] {$E$}
		(nand2.in 1) |- (dot)

		(3,1) node [american nand port] (nor1) {}
	%	(nor1)+(0,0.25) |- ++(-0.7,1) node [ocirc] () {} node [anchor=south] {$B2$}
		(3,-1.5) node [american nand port] (nor2) {}
	%	(nor2)+(0,-0.25) |- ++(-0.7,-1) node [ocirc] () {} node [anchor=north] {$B1$}
%		(not1.out) -- (not2.in)
	%	(not2.out) -| ++(0.5,0.5)  coordinate (a-a) to [kinky cross=(a-a)--(a-b), kinky crosses=left] ++(-3,1)  coordinate (a-b) |- (not1.in)
		(nor1.out)  ++(0.5,-0.5)  coordinate (a-a) %coords of the crossing wire
		(nor2.in 2)  ++(-1.5,0.5)  coordinate (a-b)

		(nor1.in 2)++(-1.135,-0.225) node (in) {} % end of the wire with kinky bump
		(nor2.out)-| ++(0.5,0.5) to  [kinky cross=(a-a)--(a-b), kinky crosses=left] (in)
		(nor1.in 2)-| ++(-1,-0.3)
		(nor1.out) -- ++(1.5,0) node [ocirc] () {} node [anchor=west] {$Q$}

		(nor1.out) -| ++(0.5,-0.5) -- ++(-3.043,-1.5) |- (nor2.in 1)
		(nor2.out) -- ++(1.5,0) node [ocirc] () {} node [anchor=west] {$\overline{Q}$}
		(nor1.out) |-  ++(0.25,0)
		%(nor2.out) |-  ++(0.25,0) node [circ] () {} node [anchor=south] {$Q$}

		%(nor1.in 1) -- ++(-2,0) node [ocirc] () {} node [anchor=east] {$\overline{S}$}
		%(nor2.in 2) -- ++(-2,0) node [ocirc] () {} node [anchor=east] {$\overline{R}$}
		(nor1.in 1) -| (nand1.out)
		(nor2.in 2) -| (nand2.out)

	;
%	\draw [dashed](-2.75,-2.25) rectangle (1,1.75);
	\end{circuitikz}
\end{center}


\begin{Q}
	Mettre en évidence l'avantage de l'entrée enable du point de vue des modifications des entrées R et S.
	\label{Q:dl1}
	\reponse{}%R
\end{Q}
\subsection{D latch}

Le D latch peut être réalisé avec le schéma suivant :
%schéma

\begin{center}
\begin{circuitikz} \draw
		(0,1.28) node [american nand port] (nand1) {}
		(0,-1.5-0.28) node [american nand port] (nand2) {}

		(nand1.in 1) -- ++(-2.5,0) node [ocirc] (D) {} node [anchor=east] {$D$}
%		(nand2.in 2) -- ++(-1.5,0) node [ocirc] () {} node [anchor=east] {$S$}
		(-2.25,-2.07) node [american not port] (not) {}
		(D) -| (not.in)
		(not.out) --	(nand2.in 2)
%		(nand1.in 2)
		(D)++(0.94,0) coordinate (Dvert) %pour avoir seulement le segment vertical pour calculer l'intersection
		(nand1.in 2) |- ++(-1.25,-1.28) coordinate (dot)
		(dot) to [kinky cross=(Dvert)--(not.in), kinky crosses=left] ++(-1.25,0)node [ocirc] () {} node [anchor=east] {$E$}
		(nand2.in 1) |- (dot)

		(3,1) node [american nand port] (nor1) {}
	%	(nor1)+(0,0.25) |- ++(-0.7,1) node [ocirc] () {} node [anchor=south] {$B2$}
		(3,-1.5) node [american nand port] (nor2) {}
	%	(nor2)+(0,-0.25) |- ++(-0.7,-1) node [ocirc] () {} node [anchor=north] {$B1$}
%		(not1.out) -- (not2.in)
	%	(not2.out) -| ++(0.5,0.5)  coordinate (a-a) to [kinky cross=(a-a)--(a-b), kinky crosses=left] ++(-3,1)  coordinate (a-b) |- (not1.in)
		(nor1.out)  ++(0.5,-0.5)  coordinate (a-a) %coords of the crossing wire
		(nor2.in 2)  ++(-1.5,0.5)  coordinate (a-b)

		(nor1.in 2)++(-1.135,-0.225) node (in) {} % end of the wire with kinky bump
		(nor2.out)-| ++(0.5,0.5) to  [kinky cross=(a-a)--(a-b), kinky crosses=left] (in)
		(nor1.in 2)-| ++(-1,-0.3)
		(nor1.out) -- ++(1.5,0) node [ocirc] () {} node [anchor=west] {$Q$}

		(nor1.out) -| ++(0.5,-0.5) -- ++(-3.043,-1.5) |- (nor2.in 1)
		(nor2.out) -- ++(1.5,0) node [ocirc] () {} node [anchor=west] {$\overline{Q}$}
		(nor1.out) |-  ++(0.25,0)
		%(nor2.out) |-  ++(0.25,0) node [circ] () {} node [anchor=south] {$Q$}

		%(nor1.in 1) -- ++(-2,0) node [ocirc] () {} node [anchor=east] {$\overline{S}$}
		%(nor2.in 2) -- ++(-2,0) node [ocirc] () {} node [anchor=east] {$\overline{R}$}
		(nor1.in 1) -| (nand1.out)
		(nor2.in 2) -| (nand2.out)

	;
	\draw [dashed](-3.5,-3.25) rectangle (4.25,2.75);
	\end{circuitikz}
\end{center}


L'inverseur en tête du montage a pour effet de supprimer l'état illicite du bistable R-S de sortie.

\begin{Q}
	 Déterminer son fonctionnement (table de vérité simplifiée).

	\label{Q:dlatch}
	\reponse{}%R
\end{Q}
\begin{Q}
	Réaliser le D LATCH à l'aide de portes logiques. Le circuit 74HC04 contient 6 inverseurs.
	\label{Q:1}
	\reponse{}%R
\end{Q}



\begin{Q}
	Vérifier que votre circuit reproduit bien le fonctionnement attendu.
	\label{Q:1}
	\reponse{}%R
\end{Q}

\begin{Q}
	Peut-on, à votre avis, modifier D n'importe quand ?
	\label{Q:1}
	\reponse{}%R
\end{Q}

{\color{white} C'était pénible de faire tous les schémas, j'espère que vous les trouvez beaux. Si c'est pas le cas, allez voir au pôle Sud si j'y suis.}

\appendix
\section{Brochages}
%\subsection{74HC00 : 4$\times$2-NAND}

\begin{center}
		\begin{circuitikz}[scale=0.8] \draw
		(4,2.5) node [anchor=center] {$74HC00, 4\cdot NAND$}
		(2.8,1.5) node [american nand port,scale=0.8] (nand1) {}
		(1,0) node (in11) {}
		(2,0) node (in12) {}
		(3,0) node (out1) {}
		(in11) |- (nand1.in 1)
		(in12) |- ++(-0.6,0.75)|- (nand1.in 2)
		(out1) |- (nand1.out)

		(2.8+3,1.5) node [american nand port,scale=0.8] (nand2) {}
		(1+3,0) node (in21) {}
		(2+3,0) node (in22) {}
		(3+3,0) node (out2) {}
		(in21) |- (nand2.in 1)
		(in22) |- ++(-0.6,0.75)|- (nand2.in 2)
		(out2) |- (nand2.out)


		(1+2.8,5-1.5) node [american nand port,scale=0.8] (nand3) {}
		(1+1,5) node (in31) {}
		(2+1,5) node (in32) {}
		(3+1,5) node (out3) {}
		(in31) |- (nand3.in 2)
		(in32) |- ++(-0.6,-0.75)|- (nand3.in 1)
		(out3) |- (nand3.out)

		(1+2.8+3,5-1.5) node [american nand port,scale=0.8] (nand4) {}
		(2+3,5) node (in41) {}
		(3+3,5) node (in42) {}
		(4+3,5) node (out4) {}
		(in41) |- (nand4.in 2)
		(in42) |- ++(-0.6,-0.75)|- (nand4.in 1)
		(out4) |- (nand4.out)

		(7,0-0.25) node [anchor=north](gnd) {GND}
		(1,5+0.35) node [anchor=south](vcc) {VCC}

%		(-0.7,-0.8) node  {NAND}
%		(-3,1) node [american not port, scale=0.8] (not){}
%		(and1.in 1) -|   (not.out)

%		(not.in) |-  ++(-0.5,0) node [ocirc] (A) {} node [anchor=east] {A}
%		(and1.in 2) |- ++(-2.66,0) node [ocirc] (B) {} node [anchor=east] {B}
%		(and1.out) -- ++(1,0) node [ocirc] (Y) {} node [anchor=west] {Y}

	;
	\draw (0,0)rectangle (8,5);
	\foreach \x in {1,2,...,7} \filldraw [fill=white] (\x-0.25,-0.15) rectangle (\x+0.25,0.35) (\x,0.1) node {\x};
	\foreach \x in {1,2,...,7} \filldraw [fill=white] (\x-0.25,5-0.15) rectangle (\x+0.25,5+0.35);
	\foreach \x in {8,9,...,14} \draw (15-\x,5+0.1) node {\x};
	\draw (0,2) arc[start angle=-90, end angle=90, radius=0.5];
	\end{circuitikz}
%\end{center}
\hspace{1cm}
%\subsection{74HC32 : 4$\times$2-OR}
%\begin{center}
		\begin{circuitikz}[scale=0.8] \draw
		(4,2.5) node [anchor=center] {$74HC32, 4\cdot  OR$}
		(2.8,1.5) node [american or port,scale=0.8] (or1) {}
		(1,0) node (in11) {}
		(2,0) node (in12) {}
		(3,0) node (out1) {}
		(in11) |- (or1.in 1)
		(in12) |- ++(-0.6,0.75)|- (or1.in 2)
		(out1) |- (or1.out)

		(2.8+3,1.5) node [american or port,scale=0.8] (or2) {}
		(1+3,0) node (in21) {}
		(2+3,0) node (in22) {}
		(3+3,0) node (out2) {}
		(in21) |- (or2.in 1)
		(in22) |- ++(-0.6,0.75)|- (or2.in 2)
		(out2) |- (or2.out)


		(1+2.8,5-1.5) node [american or port,scale=0.8] (or3) {}
		(1+1,5) node (in31) {}
		(2+1,5) node (in32) {}
		(3+1,5) node (out3) {}
		(in31) |- (or3.in 2)
		(in32) |- ++(-0.6,-0.75)|- (or3.in 1)
		(out3) |- (or3.out)

		(1+2.8+3,5-1.5) node [american or port,scale=0.8] (or4) {}
		(2+3,5) node (in41) {}
		(3+3,5) node (in42) {}
		(4+3,5) node (out4) {}
		(in41) |- (or4.in 2)
		(in42) |- ++(-0.6,-0.75)|- (or4.in 1)
		(out4) |- (or4.out)

		(7,0-0.25) node [anchor=north](gnd) {GND}
		(1,5+0.35) node [anchor=south](vcc) {VCC}

	;
	\draw (0,0)rectangle (8,5);
	\foreach \x in {1,2,...,7} \filldraw [fill=white] (\x-0.25,-0.15) rectangle (\x+0.25,0.35) (\x,0.1) node {\x};
	\foreach \x in {1,2,...,7} \filldraw [fill=white] (\x-0.25,5-0.15) rectangle (\x+0.25,5+0.35);
	\foreach \x in {8,9,...,14} \draw (15-\x,5+0.1) node {\x};
	\draw (0,2) arc[start angle=-90, end angle=90, radius=0.5];
	\end{circuitikz}
\end{center}

%\subsection{74HC36 : 4$\times$2-NOR}
%\begin{center}
%		\begin{circuitikz}[scale=0.8] \draw
%		(4,2.5) node [anchor=center] {$74HC36, 4\times NOR$}
%		(2.8,1.5) node [american nor port,scale=0.8] (nor1) {}
%		(1,0) node (in11) {}
%		(2,0) node (in12) {}
%		(3,0) node (out1) {}
%		(in11) |- (nor1.in 1)
%		(in12) |- ++(-0.6,0.75)|- (nor1.in 2)
%		(out1) |- (nor1.out)
%
%		(2.8+3,1.5) node [american nor port,scale=0.8] (nor2) {}
%		(1+3,0) node (in21) {}
%		(2+3,0) node (in22) {}
%		(3+3,0) node (out2) {}
%		(in21) |- (nor2.in 1)
%		(in22) |- ++(-0.6,0.75)|- (nor2.in 2)
%		(out2) |- (nor2.out)
%
%
%		(1+2.8,5-1.5) node [american nor port,scale=0.8] (nor3) {}
%		(1+1,5) node (in31) {}
%		(2+1,5) node (in32) {}
%		(3+1,5) node (out3) {}
%		(in31) |- (nor3.in 2)
%		(in32) |- ++(-0.6,-0.75)|- (nor3.in 1)
%		(out3) |- (nor3.out)
%
%		(1+2.8+3,5-1.5) node [american nor port,scale=0.8] (nor4) {}
%		(2+3,5) node (in41) {}
%		(3+3,5) node (in42) {}
%		(4+3,5) node (out4) {}
%		(in41) |- (nor4.in 2)
%		(in42) |- ++(-0.6,-0.75)|- (nor4.in 1)
%		(out4) |- (nor4.out)
%
%		(7,0-0.25) node [anchor=north](gnd) {GND}
%		(1,5+0.35) node [anchor=south](vcc) {VCC}
%
%	;
%	\draw (0,0)rectangle (8,5);
%	\foreach \x in {1,2,...,7} \filldraw [fill=white] (\x-0.25,-0.15) rectangle (\x+0.25,0.35) (\x,0.1) node {\x};
%	\foreach \x in {1,2,...,7} \filldraw [fill=white] (\x-0.25,5-0.15) rectangle (\x+0.25,5+0.35);
%	\foreach \x in {8,9,...,14} \draw (15-\x,5+0.1) node {\x};
%	\draw (0,2) arc[start angle=-90, end angle=90, radius=0.5];
%	\end{circuitikz}
%\end{center}

%\subsection{74HC02 : 4$\times$2-NOR}
%\begin{center}
%		\begin{circuitikz}[scale=0.8] \draw
%		(4,2.5) node [anchor=center] {$74HC02, 4\cdot  NOR$}
%		(1.2,1.5) node [american nor port,scale=0.8, xscale=-1] (nor1) {}
%		(3,0) node (in11) {}
%		(2,0) node (in12) {}
%		(1,0) node (out1) {}
%		(in11) |- (nor1.in 1)
%		(in12) |- ++(0.6,0.75)|- (nor1.in 2)
%		(out1) |- (nor1.out)
%
%		(1.2+3,1.5) node [american nor port,scale=0.8, xscale=-1] (nor2) {}
%		(3+3,0) node (in21) {}
%		(2+3,0) node (in22) {}
%		(1+3,0) node (out2) {}
%		(in21) |- (nor2.in 1)
%		(in22) |- ++(0.6,0.75)|- (nor2.in 2)
%		(out2) |- (nor2.out)
%
%
%		(1+1.2,5-1.5) node [american nor port,scale=0.8,xscale=-1] (nor3) {}
%		(3+1,5) node (in31) {}
%		(2+1,5) node (in32) {}
%		(1+1,5) node (out3) {}
%		(in31) |- (nor3.in 2)
%		(in32) |- ++(0.6,-0.75)|- (nor3.in 1)
%		(out3) |- (nor3.out)
%
%		(1+1.2+3,5-1.5) node [american nor port,scale=0.8,xscale=-1] (nor4) {}
%		(4+3,5) node (in41) {}
%		(3+3,5) node (in42) {}
%		(2+3,5) node (out4) {}
%		(in41) |- (nor4.in 2)
%		(in42) |- ++(0.6,-0.75)|- (nor4.in 1)
%		(out4) |- (nor4.out)
%
%		(7,0-0.25) node [anchor=north](gnd) {GND}
%		(1,5+0.35) node [anchor=south](vcc) {VCC}
%
%	;
%	\draw (0,0)rectangle (8,5);
%	\foreach \x in {1,2,...,7} \filldraw [fill=white] (\x-0.25,-0.15) rectangle (\x+0.25,0.35) (\x,0.1) node {\x};
%	\foreach \x in {1,2,...,7} \filldraw [fill=white] (\x-0.25,5-0.15) rectangle (\x+0.25,5+0.35);
%	\foreach \x in {8,9,...,14} \draw (15-\x,5+0.1) node {\x};
%	\draw (0,2) arc[start angle=-90, end angle=90, radius=0.5];
%	\end{circuitikz}
%\end{center}

%\begin{center}
%		\begin{circuitikz}[scale=0.8] \draw
%		(4,2.5) node [anchor=center] {$74HC02, 4\cdot  NOR$}
%		(1.2,1.5) node [american not port,scale=0.8, xscale=-1] (nor1) {}
%		%(3,0) node (in11) {}
%		(2,0) node (in12) {}
%		(1,0) node (out1) {}
%		(in11) |- (nor1.in 1)
%		(in12) |- ++(0.6,0.75)|- (nor1.in 2)
%		(out1) |- (nor1.out)
%
%		(1.2+3,1.5) node [american not port,scale=0.8, xscale=-1] (nor2) {}
%		(3+3,0) node (in21) {}
%		(2+3,0) node (in22) {}
%		(1+3,0) node (out2) {}
%		(in21) |- (nor2.in 1)
%		(in22) |- ++(0.6,0.75)|- (nor2.in 2)
%		(out2) |- (nor2.out)
%
%
%		(1+1.2,5-1.5) node [american not port,scale=0.8,xscale=-1] (nor3) {}
%		(3+1,5) node (in31) {}
%		(2+1,5) node (in32) {}
%		(1+1,5) node (out3) {}
%		(in31) |- (nor3.in 2)
%		(in32) |- ++(0.6,-0.75)|- (nor3.in 1)
%		(out3) |- (nor3.out)
%
%		(1+1.2+3,5-1.5) node [american not port,scale=0.8,xscale=-1] (nor4) {}
%		(4+3,5) node (in41) {}
%		(3+3,5) node (in42) {}
%		(2+3,5) node (out4) {}
%		(in41) |- (nor4.in 2)
%		(in42) |- ++(0.6,-0.75)|- (nor4.in 1)
%		(out4) |- (nor4.out)
%
%		(7,0-0.25) node [anchor=north](gnd) {GND}
%		(1,5+0.35) node [anchor=south](vcc) {VCC}
%
%	;
%	\draw (0,0)rectangle (8,5);
%	\foreach \x in {1,2,...,7} \filldraw [fill=white] (\x-0.25,-0.15) rectangle (\x+0.25,0.35) (\x,0.1) node {\x};
%	\foreach \x in {1,2,...,7} \filldraw [fill=white] (\x-0.25,5-0.15) rectangle (\x+0.25,5+0.35);
%	\foreach \x in {8,9,...,14} \draw (15-\x,5+0.1) node {\x};
%	\draw (0,2) arc[start angle=-90, end angle=90, radius=0.5];
%	\end{circuitikz}
%\end{center}

\begin{center}
		\begin{circuitikz}[scale=0.8] \draw
		(4,2.5) node [anchor=center] {$74HC04, 6\cdot NOT$}
		(1.5,1) node [american not port,scale=0.55] (not1) {}
		(1,0) node (in11) {}
		%(2,0) node (in12) {}
		(2,0) node (out1) {}
		(in11) |- (not1.in)
		%(in12) |- ++(-0.6,0.75)|- (not1.in 2)
		(out1) |- (not1.out)

		(1.5+2,1) node [american not port,scale=0.55] (not2) {}
		(1+2,0) node (in21) {}
		%(2+2,0) node (in22) {}
		(2+2,0) node (out2) {}
		(in21) |- (not2.in)
		%(in22) |- ++(-0.6,0.75)|- (not2.in 2)
		(out2) |- (not2.out)

		(1.5+4,1) node [american not port,scale=0.55] (not5) {}
		(1+4,0) node (in51) {}
		%(2+2,0) node (in22) {}
		(2+4,0) node (out5) {}
		(in51) |- (not5.in)
		%(in22) |- ++(-0.6,0.75)|- (not2.in 2)
		(out5) |- (not5.out)

		(1+1.5,5-1) node [american not port,scale=0.55] (not3) {}
		(1+1,5) node (in31) {}
		%(2+1,5) node (in32) {}
		(2+1,5) node (out3) {}
		(in31) |- (not3.in)
		%(in32) |- ++(-0.6,-0.75)|- (not3.in 1)
		(out3) |- (not3.out)

		(1+1.5+2,5-1) node [american not port,scale=0.55] (not4) {}
		(2+2,5) node (in41) {}
		%(3+2,5) node (in42) {}
		(3+2,5) node (out4) {}
		(in41) |- (not4.in)
		%(in42) |- ++(-0.6,-0.75)|- (not4.in 1)
		(out4) |- (not4.out)

		(1+1.5+4,5-1) node [american not port,scale=0.55] (not6) {}
		(2+4,5) node (in61) {}
		%(3+2,5) node (in42) {}
		(3+4,5) node (out6) {}
		(in61) |- (not6.in)
		%(in42) |- ++(-0.6,-0.75)|- (not4.in 1)
		(out6) |- (not6.out)

		(7,0-0.25) node [anchor=north](gnd) {GND}
		(1,5+0.35) node [anchor=south](vcc) {VCC}

%		(-0.7,-0.8) node  {NAND}
%		(-3,1) node [american not port, scale=0.8] (not){}
%		(and1.in 1) -|   (not.out)

%		(not.in) |-  ++(-0.5,0) node [ocirc] (A) {} node [anchor=east] {A}
%		(and1.in 2) |- ++(-2.66,0) node [ocirc] (B) {} node [anchor=east] {B}
%		(and1.out) -- ++(1,0) node [ocirc] (Y) {} node [anchor=west] {Y}

	;
	\draw (0,0)rectangle (8,5);
	\foreach \x in {1,2,...,7} \filldraw [fill=white] (\x-0.25,-0.15) rectangle (\x+0.25,0.35) (\x,0.1) node {\x};
	\foreach \x in {1,2,...,7} \filldraw [fill=white] (\x-0.25,5-0.15) rectangle (\x+0.25,5+0.35);
	\foreach \x in {8,9,...,14} \draw (15-\x,5+0.1) node {\x};
	\draw (0,2) arc[start angle=-90, end angle=90, radius=0.5];
	\end{circuitikz}
\end{center}

Pour information, les documentations complètes :
\begin{itemize}
\item 74HC00 (Quad NAND) \href{https://www.fairchildsemi.com/datasheets/74/74VHC00.pdf}{https://www.fairchildsemi.com/datasheets/74/74VHC00.pdf}
\item 74HC04 (Hex NOT) \href{https://www.fairchildsemi.com/datasheets/74/74VHC04.pdf}{https://www.fairchildsemi.com/datasheets/74/74VHC04.pdf}
\item 74HC32 (Quad OR) \href{https://www.fairchildsemi.com/datasheets/74/74VHC32.pdf}{https://www.fairchildsemi.com/datasheets/74/74VHC32.pdf}
\end{itemize}
%liens doc

\end{document}
