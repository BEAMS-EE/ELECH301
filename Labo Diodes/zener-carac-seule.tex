\documentclass{standalone}
\usepackage[utf8x]{inputenc}

\usepackage[frenchb]{babel}
\usepackage[T1]{fontenc}

\usepackage{graphicx}
\usepackage{amssymb}
\usepackage{amsmath}
\usepackage{wasysym} %smiley
\usepackage{hyperref}% hyperliens
\usepackage{tikz}
\usetikzlibrary{babel,positioning,calc}
\usepackage{textcomp}
% \usepackage{minted}
\usepackage[long]{datetime}
\usepackage{gensymb} % \ohm, celsius
\usepackage{framed}
\usepackage{pdfpages}
\usepackage{todo}
\usepackage{paralist}
\usepackage{multicol}

\usepackage{mathastext} % math as standfard text : units are respecting typography conventions.
\usepackage{fancyhdr} %en-tête
\usepackage{qrcode}




\usepackage{xspace} % typographie IN
\usepackage{hyperref}% hyperliens
\usepackage[all]{hypcap} %lien pointe en haut des figures
\usepackage[french]{varioref} %voir x p y
\usepackage{fancyhdr}% en têtes
%\input cyracc.def
\usepackage{pgfplots}
\usetikzlibrary{babel,positioning,calc}
\usepackage[americanresistors ]{circuitikz}
%\usepackage[]{gnuplottex}
\usepackage{ifthen}

\usepackage[]{pdfpages}
\usepackage[]{subfig}
\usepackage[]{attachfile}


\setlength{\parskip}{0.5cm plus4mm minus3mm} %espacement entre §
\setlength{\parindent}{0pt}


\begin{document}
\begin{tikzpicture}
	\begin{axis}[ %title ={4Hz Sine  Wave},
	% width=7cm,
	% height=5cm,
	axis lines=middle,
	% ymin=-10,
	ymax=4,
	xlabel ={$V_z$},
	xticklabels={},
	yticklabels={},
	% ytick={-10,-8,-6,-5,-4,-3,-2,-1,1,2,3,4,5,6,8,10}
	ylabel ={$I_z$},
    % grid=both,
    % grid style={line width=.1pt, draw=black!60},
    % major grid style={line width=.2pt,draw=black},
    % ultra thick,
    % minor tick num=5,
    % enlargelimits={abs=0.5},
    % axis line style={latex-latex},
    yticklabel style={font=\normalsize,fill=white},
    xlabel style={at={(ticklabel* cs:1)},anchor=north west},
    % ylabel style={at={(ticklabel* cs:1)},anchor=south west},
	]
	\addplot[%
	% red,
	domain=1:3,
	thick,
	samples=100
	]
	{0.9*(x-1)^2};
	\addplot[%
	% red,
	domain=0:1,
	thick,
	samples=100
	]
	{0};
	\end{axis}
	\end{tikzpicture}
\end{document}
