\documentclass{standalone}
\usepackage[utf8x]{inputenc}

\usepackage[frenchb]{babel}
\usepackage[T1]{fontenc}

\usepackage{graphicx}
\usepackage{amssymb}
\usepackage{amsmath}
\usepackage{wasysym} %smiley
\usepackage{hyperref}% hyperliens
\usepackage{tikz}
\usetikzlibrary{babel,positioning,calc}
\usepackage{textcomp}
% \usepackage{minted}
\usepackage[long]{datetime}
\usepackage{gensymb} % \ohm, celsius
\usepackage{framed}
\usepackage{pdfpages}
\usepackage{todo}
\usepackage{paralist}
\usepackage{multicol}

\usepackage{mathastext} % math as standfard text : units are respecting typography conventions.
\usepackage{fancyhdr} %en-tête
\usepackage{qrcode}




\usepackage{xspace} % typographie IN
\usepackage{hyperref}% hyperliens
\usepackage[all]{hypcap} %lien pointe en haut des figures
\usepackage[french]{varioref} %voir x p y
\usepackage{fancyhdr}% en têtes
%\input cyracc.def
\usepackage{pgfplots}
\usetikzlibrary{babel,positioning,calc}
\usepackage[americanresistors ]{circuitikz}
%\usepackage[]{gnuplottex}
\usepackage{ifthen}

\usepackage[]{pdfpages}
\usepackage[]{subfig}
\usepackage[]{attachfile}


\setlength{\parskip}{0.5cm plus4mm minus3mm} %espacement entre §
\setlength{\parindent}{0pt}


\begin{document}
\begin{tikzpicture}
	\begin{axis}[ %title ={4Hz Sine  Wave},
	% width=7cm,
	% height=5cm,
	axis lines=middle,
	% ymin=-10,
	ymax=1.5,
	xlabel ={$I_{out}$},
	xticklabels={},
	yticklabels={},
	% ytick={-10,-8,-6,-5,-4,-3,-2,-1,1,2,3,4,5,6,8,10}
	ylabel ={$V_{out}$},
    % grid=both,
    % grid style={line width=.1pt, draw=black!60},
    % major grid style={line width=.2pt,draw=black},
    % ultra thick,
    % minor tick num=5,
    % enlargelimits={abs=0.5},
    % axis line style={latex-latex},
    yticklabel style={font=\normalsize,fill=white},
    xlabel style={at={(ticklabel* cs:1)},anchor=north west},
    % ylabel style={at={(ticklabel* cs:1)},anchor=south west},
	]
	\addplot[%
	% red,
	domain=0:2,
	thick,
	samples=100
	]
	{1};
	% \addlegendentry{$V_{in}$}
	\addplot[%
	% red,
	domain=2:3,
	thick,
	samples=100
	]
	{-x+3};
	\end{axis}
	\draw[dashed] (4.55,0) -- (4.55,5);
	\draw[decorate, decoration={brace, amplitude=5pt}] ([yshift=-0.2cm]4.55,0)-- node[below=0.25cm, text width=4cm]
         {Grosse charge, $I_{out}$ est donc faible et la Zener est en avalanche. La charge est régulée.}([yshift=-0.2cm]0,0); % Pour avoir une accolade avec la pointe vers le bas, d'abord donner la coordonnee de droite.
	\draw[decorate, decoration={brace, amplitude=5pt}] ([yshift=-0.2cm]6.85,0)-- node[below=0.25cm, text width=2cm]
         {Diviseur résistif, la Zener est bloquante.}([yshift=-0.2cm]4.55,0); % Pour avoir une accolade avec la pointe vers le bas, d'abord donner la coordonnee de droite.
    \draw [->] (0,4.5) to [out=10,in=170] node[above]{$R_{ch} \searrow$} (6.85,4.5);
% Note that I had to replace the – by “to”.  Notice how the angles work:
% •
% When the curves goes “out” of (0,0), you put a needle with one extremity
% on the starting point and the other one facing right and you turn it coun-
% terclockwise until it is tangent to the curve.  The angle by which you have
% to turn the needle gives you the “out” angle.
% •
% When the curves goes “in” at (2,1.5), you put a needle with one extremity
% on the arrival point and the other one facing right and you turn it coun-
% terclockwise until it is tangent to the curve.  The angle by which you have
% to turn the needle gives you the “in” angle.
% https://cremeronline.com/LaTeX/minimaltikz.pdf
% A very minimal introduction to TikZ, by Jacques Cremer
	\end{tikzpicture}
\end{document}
