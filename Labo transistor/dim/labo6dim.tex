\documentclass[11pt,a4paper]{article}

\usepackage[T1]{fontenc}
\usepackage[utf8]{inputenc}
\usepackage[]{tikz}
%\usetikzlibrary{graphdrawing}
%\usepackage{ucs}
\usepackage{amsthm} %numéroter les questions
\usepackage[frenchb]{babel}
\usepackage{datetime}
\usepackage{xspace} % typographie IN
\usepackage{hyperref}% hyperliens
\usepackage[all]{hypcap} %lien pointe en haut des figures
\usepackage[french]{varioref} %voir x p y
\usepackage{fancyhdr}% en têtes
%\input cyracc.def
\usepackage{pgfplots}
\usetikzlibrary{babel,positioning,calc}
\usepackage[]{graphicx} %include pictures
\usepackage[siunitx ]{circuitikz}
\usepackage{gnuplottex}
\usepackage{ifthen}

\usepackage[top=1.3 in, bottom=1.3 in, left=1.3 in, right=1.3 in]{geometry} % Yeah, that's bad to play with margins
\usepackage[]{pdfpages}

\usepackage[]{attachfile}

\newdateformat{mydate}{v1.0.0}%hack pour remplacer \THEYEAR

%cyr
\newcommand\textcyr[1]{{\fontencoding{OT2}\fontfamily{wncyr}\selectfont #1}}

\newboolean{corrige}
\setboolean{corrige}{true}%corrigé
%\setboolean{corrige}{false}% pas de corrigé

\newboolean{annexes}
\setboolean{annexes}{true}%annexes
%\setboolean{annexes}{false}% pas de annexes

\newboolean{mos}
%\setboolean{mos}{true}%annexes
\setboolean{mos}{false}% pas de annexes

\usepackage{aeguill} %guillemets


%\usetikzlibrary{graphs} 
\usetikzlibrary{graphs}

%% fancy header & foot
\pagestyle{fancy}
\lhead{[ELEC-H-301] Électronique appliquée\\ Labo \no 4 : ampli à transistors}
\rhead{\mydate\today\\ page \thepage}
\chead{\ifthenelse{\boolean{corrige}}{Corrigé}{}}
\cfoot{}
%%

\pdfinfo{
/Author (Yannick Allard, ULB -- BEAMS)
/Title (Labo 4 ELEC-H-301, transistor MOS)
/ModDate (D:\pdfdate)
}

\hypersetup{
pdftitle={Labo 4 [ELEC-H-301] Électronique appliquée : ampli à transistors},
pdfauthor={Yannick Allard, ©2010-2016 ULB - BEAMS  },
pdfsubject={Transistor MOS en amplification}
}

\theoremstyle{definition}% questions pas en italique
\newtheorem{Q}{Question}[] % numéroter les questions [section] ou non []

\newcommand{\reponse}[1]{% pour intégrer une réponse : \reponse{texte} : sera inclus si \boolean{corrige}
	\ifthenelse {\boolean{corrige}} {\paragraph{Réponse :} #1} {}
 }

\newcommand{\addcontentslinenono}[4]{\addtocontents{#1}{\protect\contentsline{#2}{#3}{#4}{}}}

\date{\vspace{-1cm}\mydate\today}
\title{\vspace{-2cm} Labo \no 4\\ Électronique appliquée [ELEC-H-301]\\Réalisation d'un ampli à transistor\ifthenelse{\boolean{corrige}}{~\\Corrigé}{}}

%\author{\vspace{-1cm}}%\textsc{Yannick Allard}}

\setlength{\parskip}{0.5cm plus4mm minus3mm} %espacement entre §
\setlength{\parindent}{0pt}

\begin{document}

%
\section{Synthèse -- dimensionnement d'un étage}

\textit{Cahier des charges :}
\begin{itemize}
\item Bande passante : 20Hz--20kHz
\item Gain : 26dB
\item Amplitude de l'entrée : 0.1V, sinusoïdale, moyenne nulle
\item Tension de sortie à moyenne nulle
\item Impédance d'entrée : $Z_{in}\geq10k\Omega$
\item Impédance de sortie : $Z_{out}\leq 1k\Omega$
\item Puissance statique dissipée par le transistor : $P_T\leq 0.5W$
\item Alimentation : $12V$ continue
\item Impédance de la charge : $Z_{ch}\geq 3.3k\Omega$
\end{itemize}


~\\
\textit{Démarche proposée :}
\begin{enumerate}
\item Identifier les contraintes imposées par le cahier des charges.
\item Choisir la topologie de l'amplificateur (= schéma que vous avez obtenu en répondant aux questions du labo).
\item Identifier les différents paramètres en fonction du cahier des charges (faire un schéma à petit signal).
\item Dimensionner les valeurs des différents composants du circuit sur base des courbes théoriques en annexe.
\item Déterminer la valeur de la tension de polarisation.
\item {Vérifier expérimentalement}\footnote{Ce point ne sera pas abordé dans ce document.} en tenant compte des caractéristiques réelles de votre transistor. La polarisation devra être ajustée pour obtenir le bon gain.
\end{enumerate}

\reponse{
~\\
\subsection{Contraintes :}
\begin{enumerate}
\item Gain de $26dB$ $\Longrightarrow$ $A_v= \pm 20$
\item Tension d'entrée (amplitude) : $V_{in}=0.1V$ $\Longrightarrow$ $V_{out}=2V$, moyennes nulles
\item $Z_{ch}\geq 3.3k\Omega$, $Z_{in}\geq10k\Omega$, $Z_{out}\leq 1k\Omega$
\item $E=12V$
\item $P_T=P_{\mbox{transistor}}\leq 0.5W$ il s'agit de la puissance statique dissipée par le transistor (et uniquement par le transistor)
\item $BW=20 \rightarrow 20kHz$
\end{enumerate}

\subsection{Topologie :}

On utilise le schéma de l'amplificateur inverseur à NMOS (inverseur) du labo :
\begin{center}
	\begin{circuitikz}[scale=1]\draw
	(0,1) to [short,o-] (9,1)
	(4,6) to [short] (9,6)
	(0,3) node[anchor=east] {In} to [short,o-] (1,3)
	(0,3) to [open, v_=$V_{in}$]  (0,1)
	(1,3) to [C=$C_{in}$ ](1.5,3)
	(1.5,3) to [short,-*] (2,3)
%
	(2,6) node[alimp ] (alim) {}
	(alim.text) node {$+V_{dc}$}
%		
	(2,3) to [R, l_=$R_{b1}$](2,6)
	(2,3) to [R=$R_{b2}$](2,1)
%		
	(4,3) node[nigfetec] (mos) {}
	%(mos.B) node[anchor=west] {B}
	(mos.G) to [short] (2,3)
%		
	(mos.D) to (4,4) to [R, l_=$R_D$] (4, 6)		
	%(4,5.5) to [R] (mos.D)
%		
	(mos.D) to [short,-*](4,3.5)  to [short] (4.25,3.5)
%		
	(mos.S) to [short] (4,1)% to [short, -o](2,0)  node[anchor=west] {S}
%		
	(4.25,3.5) to [C, l^=$C{out}$] (6,3.5) to  [short](6,3.5) to [short,-o](6.5,3.5)node [anchor=south] {Out}	
	(6,3.5) to [generic, l_=$Z_{ch}$] (6,1)
	(6.5,3.5) to [open,v^=$V_{out}$] (6.5,1)
%		
	(9,1) to [battery, l_=$E$](9,6)
	;\end{circuitikz}
	\end{center}
	
	Cette topologie impose que $Av=-20$ car cet étage d'amplification est inverseur.


\subsection{Dimensionnement du gain et du point de polarisation}
Processus:
\begin{enumerate}
\item Dimensionnement du gain à vide
\item Choix d'un point de polarisation
\item Influence de la charge sur le gain
\end{enumerate}

On connaît le gain de cet étage\textbf{ à vide }lorsqu'il est polarisé correctement, il suffit d'analyser le schéma équivalent à petits signaux (voir labo) :

$$V_{out}=-g_m\cdot  R_D \cdot V_{in}$$

d'où : $$A_v=-g_m\cdot  R_D =-20$$

%L'influence de la charge sera étudiée à la fin de cette partie.

La limite en puissance $P_T=\frac{E}{2}\cdot I_D\leq0.5W$ (\textit{cf} Q.1 du labo) impose $I_D\leq 83mA$ avec $E=12V$. (Attention : seule la consommation statique du transistor est prise en compte ici)

On obtient alors $g_m\leq0.21S$ pour $I_D\leq83mA$ (voir courbes).

En pratique, on peut imposer $I_D$, $g_m$ ou $R_D$, mais il faut veiller à ce que la charge ait une faible influence sur le gain ($R_{D}\ll R_{ch}$). Pour un même gain, il est donc intéressant de choisir $R_D$ la plus faible possible, ce qui implique que $g_m$ sera plus grand et donc $I_D$ également. Le choix de $I_D$ n'est plus contraint que par la puissance dissipée par le transistor (et également le gain).

Si on impose $I_D=40mA<83mA$, $g_m=0.162S$ (voir courbes).

Alors on a $R_D=125\Omega$. En prenant la valeur normalisée la plus proche ($150\Omega$), on obtient $V_{DS}=6V$. % (la contrainte d'excursion de sortie de $2V$ minimum est respectée).
(L'influence de la charge a été négligée dans un premier temps)

$V_{GS}=2.95V$ d'où $V_{dc}=5.9V$ avec $R_{b1}=R_{b2}$.

En traçant la droite de charge sur le graphe $I_D=f\left(V_{DS}\right)$, on peut déterminer que l'amplitude maximum du signal de sortie vaut $\approx 5.5V$ ($>2V$ nécessaires).

Il faut vérifier que l'influence de la charge reste faible sur le gain : $R_D\ll R_{ch}$ permet de négliger $R_{ch}$ dans l'expression complète du gain $A_v=-g_m\cdot \left( R_D//R_{ch} \right)$ dans la bande passante du montage.

Le choix de $R_D=150\Omega$ implique une imprécision de $\approx 16\%$ sur le gain. %Il ne faut pas oublier que le $g_m$ est donné avec une dispersion de 100\%
Cette imprécision sera corrigée en pratique par un étalonnage du montage (qui restera très sensible à la température de toute façon).

\begin{center}
\fbox{$E=12V$}
\fbox{$R_D=150\Omega$}
\fbox{$I_D=40mA$}
\fbox{$P_T=240mW$}
\fbox{$V_{dc}=5.9V$}
\fbox{$R_{b1}=R_{b2}$}
\end{center}

\subsection{Dimensionnement des capacités, impédances d'entrée et de sortie}

\subsubsection{Impédance d'entrée}
L'analyse du schéma à petit signal permet d'exprimer : $$Z_{in}=(R_{b1}//R_{b2})+\frac{1}{jC_{in}\omega}$$

Pour toute fréquence, $\left|Z_{in}\right|\geq R_{b1}//R_{b2}$, ce qui va impliquer que $R_{b1}//R_{b2}\geq 10k\Omega$.

Imposer $R_{b1}=R_{b2}=100k\Omega$ permet de respecter les contraintes imposées (et diminuer le courant nécessaire à la polarisation de la grille, le courant injecté par la source de tension de polarisation et la valeur de $C_{in}$).

\subsubsection{Fréquence de coupure de tout le montage}
%Cette analyse n'a pas été faite au tableau pendant la séance de labo par manque de temps.

L'analyse à petit signal du montage permet de constater que le montage se comporte comme une composition de deux filtres passe-haut du premier ordre. Si chaque filtre est calculé pour avoir une fréquence de coupure de $20Hz$, la composée des deux filtres aura une fréquence de coupure de $31Hz$ car la fréquence pour laquelle la composée vaut $-3dB$ est plus élevée. Il suffit d'analyser la fonction de transfert complète ou de tracer le diagramme de Bode pour constater cet effet. Le graphe suivant montre la réponse en fréquence d'un passe-haut d'ordre 1 et la réponse de la composée de deux passe-haut de même fréquence de coupure ($20Hz$) :
%
\begin{center}
\includegraphics[width=10cm]{bode_corr_20-31.pdf}
\end{center}
%
L'étude théorique de la composition des filtres du premier ordre montre que la composée de $n$ filtres passe-haut du premier ordre ayant la \textbf{même} fréquence de coupure $f_c$ auront une fréquence de coupure qui vaudra\footnote{La démonstration de cette propriété est laissée au lecteur, il suffit d'exprimer la composée et identifier pour quelle fréquence son gain vaut $-3dB$.} $$f_{c_{\mbox{composée}}}=\frac{f_c}{\sqrt{2^{1/n}-1}}$$ Dans le cas présent, pour que la fréquence de coupure du système soit $f_{c_{\mbox{composée}}}=20Hz$, il faudra choisir $f_c=12.8Hz$.

\subsubsection{Capacité d'entrée}
On désire amplifier un signal audio, par conséquent, l'entrée du circuit doit être un filtre passe-haut de fréquence de coupure inférieure à $20Hz$ à $-3dB$. Il faut également éviter que la composante continue appliquée sur la grille implique un courant continu dans la source de tension d'entrée.

Le schéma a petits signaux du montage fait apparaître que :
$$f_{c_{in}}=\frac{1}{2\pi\cdot\left(R_{b1}//R_{b2}\right)\cdot C_{in}}\leq 12.8Hz$$

En choisissant $R_{b1}=R_{b2}=100 k\Omega$, on trouve $C_{in}\geq 248nF$ la valeur normalisée la plus proche est $270nF$ qui donne une fréquence de coupure de $11.8Hz$ aux tolérances sur les composants près.

\begin{center}
\fbox{$R_{b1}=R_{b2}=100 k\Omega$}
\fbox{$C_{in}=270nF$}
\end{center}

%Avec $C_{in}=270nF$, on obtient $f_c=11.8Hz$.


\subsubsection{Capacité de sortie}
Il s'agit de couper la composante continue ajoutée pour pouvoir amplifier, une nouvelle fois, un filtre passe-haut est nécessaire.
La charge branchée sur cet étage est inconnue à priori, seule son impédance minimale est connue,  il s'agit du pire cas.

Le schéma à petit signaux et une transformée Norton--Thévenin fait apparaître que :

$$f_{c_{out}}=\frac{1}{2\pi\cdot\left(R_{D}+R_{ch}\right)\cdot C_{out}}\leq 12.8Hz$$ 

Dans le pire cas ($R_{ch}=3.3k\Omega$), $C_{out}\geq 3.6\mu F$, qui donnera la fréquence de coupure basse maximum du filtre. Si la charge a une impédance plus grande, la fréquence de coupure sera plus faible.

\begin{center}
\fbox{$C_{out}\geq 3.9\mu F$}
\end{center}

\subsubsection{Impédance de sortie}
Le même schéma à petit signal permet d'identifier l'impédance de sortie qui vaut : $$Z_{out}=R_D+\frac{1}{jC_{out}\omega}$$

Dans toute la bande passante, on veut $Z_{out}\leq 1k\Omega$, ce qui implique $$C_{out}\geq \frac{1}{2\pi\cdot f_c \sqrt{Z_{out_{lim}}^2-R_D^2}}$$

D'où $C_{out}\geq 8.0\mu F$.

Avec l'autre contrainte sur $C_{out}$, on peut choisir $C_{out}=8.2\mu F$ (valeur normalisée).

Ce qui permet d'avoir $f_{c_{out}}\leq 5.6Hz$ et respecter la contrainte sur l'impédance de sortie et sur la fréquence de coupure.


\begin{center}
\fbox{$C_{out}= 8.2\mu F$}
\end{center}

\subsubsection{Bande passante totale du montage}

L'analyse de l'expression complète du gain en fonction de tous les éléments du circuit permet d'obtenir la fréquence de coupure à $-3dB$ du montage qui doit valoir $20Hz$ maximum. Comme il y a deux filtres, il est nécessaire de vérifier si la composée des deux respecte les contraintes, ce qui est le cas ici.

Si la bande passante n'avait pas été respectée, il aurait fallu diminuer la fréquence de coupure la plus élevée entre les deux filtres.


\subsection{Récapitulatif}

\begin{center}
\begin{tabular}{lll}
$E=12V$ & $R_D=150\Omega$ & $C_{in}=270nF$ \\ 
%\hline 
$V_{dc}=5.90V$ & $R_{b1}=R_{b2}=100 k\Omega$ & $f_{c_{in}}=11.8Hz$ \\ 
%\hline 
$V_{DS}=6.0V$ & $I_D=40mA$ & $C_{out}=8.2\mu F$ \\ 
%\hline 
$V_{GS}=2.95V$ & $f_{c_{\mbox{totale}}}=13.8Hz$ & $f_{c_{out}}=5.6Hz$ \\ 
\end{tabular} 

\end{center}

{\color{white}Je ferme ici la parenthèse laissée ouverte au labo \no 4 : )}

Le graphe des dépendances entre paramètres/contraintes est :

\begin{tikzpicture}[->, >=latex,minimum size= 1.25cm,scale=0.9]
\usetikzlibrary{calc}
{

  \node[draw, circle] (id) at (0-90:2) {$I_D$};
  \node[draw, circle] (vdsq) at (72-90:2) {$V_{DS}$};
  \node[draw, circle, color=red] (vout) at (144-90:2) {$v_{out}$};
   \node[draw, circle, color=red] (av) at (-144-90:2) {$A_{v}$};
    \node[draw, circle] (gm) at (-72-90:2) {$g_m$};
    
    
%    /pgf/arrow keys/length= 10mm;
  \draw   (id) to (vdsq);
  \draw (vdsq) -- (vout) ;
   \draw (vout) to (av);
   \draw(av) to (gm);
   \draw(gm) to (id);
  \draw (gm) to (av);
   \draw(av) to (vout);
 \draw  (id) to (gm);
 
 \coordinate (in) at ($(72-90:2)+(20:4)+(0.5,0)$){};
% (in)++(1,1);
  \node[draw, circle] (vdc) at (in) {$V_{dc}$};
  \node[draw, circle] (rb1) at ($(in)+(135:2) $) {$Rb_1$};
  \node[draw, circle] (rb2) at ($(in)+(-45:2) $) {$Rb_2$};
   \node[draw, circle, color=red] (zin) at ($(in)+(45:2) $) {$Z_{in}$};
   \node[draw, circle] (cin) at ($(in)+(45:4) $) {$C_{in}$};
   \node[draw, circle, color=red] (fcin) at ($(in)+(45:7) $) {$f_{C_{in}}$};
   
   \node[draw, circle] (vgsq) at ($(in)+(45:-2) $) {$V_{GS}$};
  
  \draw (id) to (vgsq);
  \draw (vgsq) to (id);
  \draw (vgsq) to (vdc);
  \draw (rb1) to (vdc);
  \draw (rb2) to (vdc);
  \draw (zin) to (rb1);
  \draw (zin) to (rb2);
  \draw (zin) to (cin);
  \draw (fcin) to (cin);
  \draw (fcin) to (rb1);
  \draw (fcin) to (rb2);
  
	\coordinate (out) at ($(av)+(105:2)$){};
% (in)++(1,1);
  \node[draw, circle, color=red] (zch) at ($(out)+(-30+15:2)$) {$Z_{ch}$};
  \node[draw, circle, color=red] (fcout) at ($(out)+(30+15:2)$) {$f_{c_{out}}$};
  \node[draw, circle] (cout) at ($(out)+(90+15:2)$) {$C_{out}$};
   \node[draw, circle, color=red] (zout) at ($(out)+(150+15:2)$) {$Z_{out}$};
   \node[draw, circle] (rd) at ($(out)+(210+15:2)$) {$R_{D}$};
%   \node[draw, circle] (fcin) at ($(out)+(-30:3)$) {$f_{C_{in}}$};
   
 %  \node[draw, circle] (vgsq) at ($(in)+(45:-2) $) {$V_{GSQ}$};
  
  \draw (av) to (rd);
  \draw (rd) to (fcout);
  \draw (zout) to (rd);
  \draw (zout) to (cout);
  \draw (fcout) to (cout);
  \draw (zch) to (fcout);
  \draw (zch) to (av);
  
  \draw (av) to (vgsq);
  
  \coordinate (pow) at ($(id)$){};
% (in)++(1,1);
  \node[draw, circle, color=red] (e) at ($(pow)+(-135:2)$) {$E$};
  \node[draw, circle, color=red] (pt) at ($(pow)+(-45:2)$) {$P_T$};
  \draw (e) to (id);
  \draw (pt) to (id);
  
  \node[draw, circle, color=red] (cont) at (-2,-6) {contrainte};
  \node[draw, circle] (param) at (6,-6) {paramètre};
  
	\node[draw, circle, color=red] (vin) at ($(vout)+(45:2)$) {$v_{in}$};  
	\draw (vin) to (vout);
  
  \draw (cont) to node [midway, above=-2mm] {influe sur/limite}(param);
}
\end{tikzpicture}

\newpage
\begin{center}
\includegraphics[width=15cm]{courbes_mos_2k16-crop_corr.pdf}
\end{center}
}%réponse
\ifthenelse{\not \boolean{annexes}}{\end{document}}{}
\clearpage
\appendix
\newgeometry{top=1 in, bottom=1 in, left=1 in, right=1 in} % Yeah, that's bad to play with margins


\end{document}